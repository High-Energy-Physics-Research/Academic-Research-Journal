\section{Tasks Status} % Seções são adicionadas para organizar sua apresentação em blocos discretos, todas as seções e subseções são automaticamente exibidas no índice como uma visão geral da apresentação, mas NÃO são exibidas como slides separados.

%----------------------------------------------------------------------------------------

\begin{frame}
  \frametitle{Tasks Status}

  % Ajustes úteis em tabelas no Beamer
  \centering
  \renewcommand{\arraystretch}{1.15} % mais espaço vertical entre linhas

  \begin{table}
    \begin{tabular}{|p{0.58\textwidth}|c|p{0.22\textwidth}|}
      \hline
      \textbf{Task} & \textbf{Status} & \textbf{Link} \\
      \hline

      \textbf{Escrever projeto} &  &  \\
      \hline
      \hspace{1.2em}\small$\hookrightarrow$ Definir escopo e objetivos & \done &  \\
      \hline
      \hspace{1.2em}\small$\hookrightarrow$ Esboçar introdução (motivação + observáveis) & \done & \href{https://www.overleaf.com/project/696e133a1101ac682ea3dd98}{Projeto} \\
      \hline
      \hspace{1.2em}\small$\hookrightarrow$ Listar método/dados (toy model, GP, etc.) & \pending &  \\
      \hline
      \hspace{1.2em}\small$\hookrightarrow$ Montar bibliografia mínima & \done &  \\
      \hline

    \end{tabular}
  \end{table}
\end{frame}
%----------------------------------------------------------------------------------------


\begin{frame}
  \frametitle{Tasks Status}

  % Ajustes úteis em tabelas no Beamer
  \centering
  \renewcommand{\arraystretch}{1.15} % mais espaço vertical entre linhas

  \begin{table}
    \begin{tabular}{|p{0.58\textwidth}|c|p{0.22\textwidth}|}
      \hline
      \textbf{Task} & \textbf{Status} & \textbf{Link} \\
      \hline
      \textbf{Criar toy model} &  &  \\
      \hline
      \hspace{1.2em}\small$\hookrightarrow$ Gerar dados sintéticos (correlacionados + ruído) & \done &  \\
      \hline
      \hspace{1.2em}\small$\hookrightarrow$ Ajustar GP (kernel, hiperparâmetros) & \pending &  \\
      \hline
      \hspace{1.2em}\small$\hookrightarrow$ Validar: resíduos, log-likelihood, cobertura & \pending &  \\
      \hline
      \hspace{1.2em}\small$\hookrightarrow$ Criar 2--3 plots “de apresentação” & \pending &  \\
      \hline
    \end{tabular}
  \end{table}
\end{frame}
%----------------------------------------------------------------------------------------

% \begin{frame}{Uma imagem}
%     \frametitle{Historical Records}
%     In addition, it is possible to retrieve the full search history by running the script directly via GitHub Actions.
%    \begin{figure}[h]
%        \centering
%        \includegraphics[width=0.3\textwidth]{img/github_topic_history.png}
%        \caption{GitHub Actions interface illustrating how the workflow can be executed to retrieve the full search history for a specific topic.}
%        \label{fig:github_topic_history}
%    \end{figure}
% \end{frame}
%----------------------------------------------------------------------------------------

% \begin{frame}
% 	\frametitle{Texto em tópicos}
%      Lorem ipsum dolor sit amet, consectetur adipiscing elit:
%     \begin{itemize}
%         \item Lorem ipsum dolor sit amet.
%         \item Lorem ipsum dolor sit amet.
%     \end{itemize}
	
% \end{frame}


